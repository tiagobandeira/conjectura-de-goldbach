\documentclass[a4paper,11pt]{article}

% Pacotes básicos
\usepackage[utf8]{inputenc} % Codificação
\usepackage[T1]{fontenc}    % Fontes com acentuação
\usepackage[brazil,english]{babel}  % Idioma
\usepackage{amsmath, amssymb, amsthm} % Matemática avançada
\usepackage{geometry}       % Margens
\usepackage{graphicx}       % Imagens
\usepackage{hyperref}       % Links
\usepackage{lmodern}        % Fonte melhorada
\usepackage{mathrsfs}       % Fontes matemáticas adicionais
\usepackage{enumitem}       % Listas personalizadas

% Configurações de página
\geometry{
	a4paper,
	left=30mm,
	right=30mm,
	top=25mm,
	bottom=25mm,
}

% Comandos para título
\title{\textbf{Uma Proposta de Demonstração para a Conjectura de Goldbach Utilizando o Teorema de Dirichlet}\\
	\large (Preprint)}
\author{
	Tiago Bandeira\thanks{tiagobandeirabarros@gmail.com} 
}
\date{\today}

% Teoremas e definições
\newtheorem{theorem}{Teorema}[section]
\newtheorem{lemma}[theorem]{Lema}
\newtheorem{proposition}[theorem]{Proposição}
\newtheorem{corollary}[theorem]{Corolário}
\theoremstyle{definition}
\newtheorem{definition}[theorem]{Definição}
\theoremstyle{remark}
\newtheorem{remark}[theorem]{Observação}
\newtheorem{example}[theorem]{Exemplo}
\newtheorem{conjecture}{Conjectura}

% Início do documento
\begin{document}
	
	\maketitle
	\begin{otherlanguage}{brazil}
	\begin{abstract}
		Este trabalho apresenta uma abordagem para a Conjectura de Goldbach, utilizando conceitos da Teoria dos Números relacionados a progressões aritméticas e simetrias entre números ímpares. Por meio da definição de uma sequência \(S_n(p)\), associada a cada número primo \(p > 2\), e da aplicação do Teorema de Dirichlet, mostra-se que todo número par maior que dois pode ser representado como a soma de um primo fixo e outro primo pertencente a essa sequência. O desenvolvimento teórico inclui definições formais, proposições, corolários e um teorema central que estabelece que a união dos subconjuntos formados por somas do tipo \(p + S_n(p)\) cobre integralmente o conjunto dos números pares maiores que dois. Embora este trabalho não afirme se tal resultado configura ou não uma demonstração definitiva da conjectura, a abordagem proposta oferece uma nova perspectiva estrutural sobre o problema e abre possibilidades para pesquisas futuras baseadas em decomposições algébricas e simétricas dos números pares.
	\end{abstract}
	\end{otherlanguage}
	

	\begin{abstract}
		This paper presents an approach to the Goldbach Conjecture based on concepts from Number Theory related to arithmetic progressions and symmetries among odd numbers. Through the definition of a sequence \(S_n(p)\), associated with each prime number \(p > 2\), and the application of Dirichlet's theorem, it is shown that every even number greater than two can be expressed as the sum of a fixed prime and another prime belonging to this sequence. The theoretical development includes formal definitions, propositions, corollaries, and a central theorem that establishes that the union of the subsets formed by sums of the type \(p + S_n(p)\) fully covers the set of even numbers greater than two. While it is not claimed here whether this result constitutes a definitive proof of the conjecture, the proposed approach offers a new structural perspective on the problem and opens possibilities for future research based on algebraic and symmetric decompositions of even numbers.	
	\end{abstract}
	
	\textbf{Palavras-chave:} conjectura de goldbach; números primos; teoria dos números.
	
	%\vspace{0.5cm}
	% sumario
	%\tableofcontents
	%\vspace{1cm}
	
	%%%%%%%%%%%%%%%%%%%%%%%%%%%%%%%%%%%%%%%
	\section{Introdução}

	A Conjectura de Goldbach, proposta em 1742 por Christian Goldbach em uma correspondência com Leonhard Euler, é um dos problemas mais antigos e famosos ainda não resolvidos da Teoria dos Números. Seu enunciado é simples e elegante: \textbf{todo número inteiro par maior que dois pode ser expresso como a soma de dois números primos}. Apesar de resistir a séculos de tentativas de demonstração, a conjectura já foi verificada computacionalmente para números muito grandes, mas uma prova geral e formal permanece desconhecida.
	
	Diversas abordagens foram desenvolvidas ao longo do tempo para tentar compreender a estrutura que envolve a distribuição dos números primos e sua relação com os números pares. Dentre os trabalhos mais relevantes, destacam-se os estudos baseados em análise matemática, combinatória aditiva, estatística dos primos e o uso de progressões aritméticas associadas ao Teorema de Dirichlet, que garante a existência de infinitos primos em certas classes aritméticas.
	
	Este trabalho apresenta uma abordagem alternativa que busca estruturar a geração dos números pares maiores que dois através da soma de dois primos, sendo um deles fixado e o outro pertencente a uma sequência específica associada ao primo fixado. Mais precisamente, explora-se uma construção baseada na sequência \(S_n(p)\), composta por números ímpares simétricos relacionados a cada primo \(p > 2\), combinada com a aplicação do Teorema de Dirichlet para garantir a infinitude de primos dentro dessas sequências.
	
	A motivação para essa abordagem surge da observação das simetrias existentes entre os números ímpares menores que um dado número par, e da possibilidade de decompor o conjunto dos pares em subconjuntos construídos por somas do tipo \(p + q\), onde \(q\) percorre a sequência \(S_n(p)\). Essa decomposição revela uma estrutura interna que organiza os pares segundo progressões aritméticas, oferecendo uma nova perspectiva sobre o problema.
	
	O presente artigo está organizado da seguinte forma: na seção de \textbf{Fundamentação Teórica} são apresentadas as definições formais, proposições, corolários e os teoremas clássicos que sustentam a construção da proposta, com destaque para o Teorema de Dirichlet e as propriedades das sequências \(S_n(p)\). Na seção de \textbf{Resultados Principais} é enunciado e demonstrado o teorema central do trabalho, que mostra como a união dos subconjuntos gerados pelas somas \(p + S_n(p)\) cobre o conjunto dos números pares maiores que dois. Na sequência, apresenta-se uma \textbf{Discussão dos Resultados}, na qual são analisadas as implicações dessa estrutura sobre a Conjectura de Goldbach, bem como as limitações e potenciais desdobramentos futuros da abordagem proposta. Por fim, a seção de \textbf{Conclusão} resume os principais achados e sugere caminhos para investigações posteriores.

	
	\subsection*{A conjectura de Goldbach}
	\begin{conjecture}[Conjectura de Goldbach]
		Todo número inteiro par maior que dois pode ser expresso como a soma de dois números primos.
		
		Formalmente, para todo \(n \in \mathbb{N}\) tal que \(n > 2\) e \(n\) é par, existe \(p, q \in \mathbb{P}\) (com \(\mathbb{P}\) sendo o conjunto dos números primos) tais que:
		
		\[
		n = p + q
		\]
		
	\end{conjecture}
	
	
	%%%%%%%%%%%%%%%%%%%%%%%%%%%%%%%%%%%%%%%
	\section{Fundamentação Teórica}
	
	\begin{proposition}
		Seja $a \in \mathbb{N}$, com $a > 2$ e $a$ par. Considere a sequência ordenada dos números ímpares estritamente menores que $a$:
		
		\[
		S = \{1, 3, 5, \dotsc, a - 1\}
		\]
		
		Então, para todo $k \in S$, vale que:
		
		\[
		a = k + (a - k)
		\]
		
		e ambos $k$ e $a - k$ pertencem a $S$, sendo simétricos em relação aos termos centrais da sequência.
	\end{proposition}
	
	
	\begin{proof}
		Observemos que os elementos de $S$ podem ser descritos pela progressão aritmética:
		
		\[
		S = \{2n - 1 \mid n = 1, 2, \dotsc, \tfrac{a}{2} \}
		\]
		
		O número total de elementos da sequência é $\tfrac{a}{2}$, inteiro pois $a$ é par.
		
		Seja um elemento arbitrário $k = 2n - 1$, com $n \in \{1, 2, \dotsc, \tfrac{a}{2}\}$. Calculamos seu par simétrico:
		
		\[
		a - k = a - (2n - 1) = (a - 2n) + 1
		\]
		
		Note que $(a - 2n)$ é par, e portanto $(a - 2n) + 1$ é ímpar. Além disso, verificamos que:
		
		\[
		a - k \leq a - 1
		\]
		\[
		a - k \geq 1
		\]
		
		Logo, $a - k \in S$.
		
		A soma do par $(k, a - k)$ resulta em:
		
		\[
		k + (a - k) = (2n - 1) + (a - (2n - 1)) = a
		\]
		
		Portanto, qualquer elemento de $S$ somado ao seu simétrico gera exatamente $a$.
		
	\end{proof}
	
	\subsection{Exemplos}
	
	Seja $a = 12$:
	
	\[
	S = \{1, 3, 5, 7, 9, 11\}
	\]
	Os pares simétricos são:
	\[
	1 + 11 = 12, \quad 3 + 9 = 12, \quad 5 + 7 = 12
	\]
	
	Seja $a = 10$:
	
	\[
	S = \{1, 3, 5, 7, 9\}
	\]
	Os pares simétricos são:
	\[
	1 + 9 = 10, \quad 3 + 7 = 10, \quad 5 + 5 = 10
	\]
	O termo central $5$ se emparelha consigo mesmo.
	
	%\subsection{Conclusão}
	
	Dessa forma, qualquer número par $a > 2$ pode ser decomposto como a soma de dois números ímpares menores que $a$, escolhidos de forma simétrica em relação aos termos centrais da sequência dos ímpares menores que $a$.
		
	\vspace{1cm}
	
	\begin{proposition}\label{prop:distancia_simetrica}
		Seja $a \in \mathbb{N}$, com $a > 2$ e $a$ par, e seja
		\[
		S = \{1, 3, 5, \dotsc, a - 1\}
		\]
		a sequência dos números ímpares menores que $a$.
		
		Sejam $b, c \in S$ dois termos simétricos em relação aos termos centrais da sequência, e seja $i$ a distância posicional de $b$ até o centro da sequência $S$. Então, a diferença entre $c$ e $b$ satisfaz:
		\[
		|c - b| =
		\begin{cases}
			4i, & \text{se } |S| \text{ é ímpar} \\
			4i + 2, & \text{se } |S| \text{ é par}
		\end{cases}
		\]
		onde $|S| = \frac{a}{2}$ é o número de elementos da sequência.
	\end{proposition}
	
	\begin{proof}
		A sequência $S$ possui $N = \frac{a}{2}$ termos, com
		\[
		S = \{2n - 1 \mid n = 1, 2, \dotsc, N\}.
		\]
		
		Considere $b$ o termo na posição $n$, com $1 \leq n \leq N$, e seu termo simétrico $c$ na posição $N - n + 1$. Assim,
		\[
		b = 2n - 1, \quad c = 2(N - n + 1) - 1 = 2N - 2n + 1.
		\]
		
		A diferença entre os termos é:
		\[
		|c - b| = |(2N - 2n + 1) - (2n - 1)| = |2N - 4n + 2| = 2|N - 2n + 1|.
		\]
		
		Definimos $i$ como a distância posicional de $b$ até o centro da sequência. Dependendo da paridade de $N$, o centro é definido como:
		
		\begin{itemize}
			\item Se $N$ é ímpar, existe um termo central único na posição $m = \frac{N+1}{2}$. Assim,
			\[
			i = |n - m| = \left| n - \frac{N+1}{2} \right|.
			\]
			Neste caso,
			\[
			|N - 2n + 1| = 2i,
			\]
			portanto,
			\[
			|c - b| = 2 \times 2i = 4i.
			\]
			
			\item Se $N$ é par, existem dois termos centrais nas posições $m_1 = \frac{N}{2}$ e $m_2 = \frac{N}{2} + 1$. Para $b$ à esquerda do centro,
			\[
			i = m_1 - n \implies 2i = |N - 2n|,
			\]
			e
			\[
			|N - 2n + 1| = 2i + 1,
			\]
			portanto,
			\[
			|c - b| = 2(2i + 1) = 4i + 2.
			\]
		\end{itemize}
		
		Assim, mostramos que a diferença entre termos simétricos na sequência $S$ segue a fórmula desejada, dependendo da paridade de $N$.
		
	\end{proof}
	
	\begin{corollary} \label{coro:par_simetrico}
		Sejam $a \in \mathbb{N}$, com $a > 2$ e $a$ par, e
		\[
		S = \{1, 3, 5, \dotsc, a - 1\}
		\]
		a sequência dos números ímpares menores que $a$.
		
		Seja $b \in S$ um termo a uma distância posicional $i$ do centro da sequência. Então, seu termo simétrico $c \in S$ satisfaz:
		
		\[
		c =
		\begin{cases}
			b + 4i, & \text{se } |S| \text{ é ímpar} \\
			b + 4i + 2, & \text{se } |S| \text{ é par}
		\end{cases}
		\]
		onde $|S| = \frac{a}{2}$ e $i$ é a distância posicional de $b$ até o centro da sequência.
	\end{corollary}
	
	
	
	\begin{theorem}[Teorema de Dirichlet para Progressões Aritméticas] \label{teo:dirichlet}
		Sejam $a$ e $d$ inteiros positivos tais que $mdc(a, d) = 1$. Então, a progressão aritmética da forma
		
		\[
		a,\, a + d,\, a + 2d,\, a + 3d,\, \dots
		\]
		
		contém infinitos números primos.
	\end{theorem}
	
	O Teorema de Dirichlet sobre progressões aritméticas garante que toda progressão da forma $a + nd$, com $mdc(a, d) = 1$, contém infinitos números primos.~Por ser complexa e envolver ferramentas matemáticas além do escopo deste trabalho a prova do Teorema de Dirichlet não será demonstrada aqui. Porém é possível encontrá-la em ~\cite{apostol}.  
	
	\begin{definition}[Sequência \(S_n(p)\)]\label{def:snp}
		Seja \(p > 2\) um número primo fixado. Definimos a sequência \(S_n(p)\), com \(n \in \mathbb{N}\), como o conjunto dos números ímpares da forma
		\[
		S_n(p) = 
		\begin{cases}
			p + 4n \\
			p + 4n + 2
		\end{cases}
		\]
		para todo \(n \in \mathbb{N}\).
	\end{definition}
	
	\vspace{0.5cm}
	
	\begin{proposition}
		A sequência \(S_n(p)\) contém infinitos números primos.
	\end{proposition}
	
	\begin{proof}
		Observamos que, nos dois casos,
		\begin{itemize}
			\item Para \(p + 4n\), temos
			\[
			p + 4n = p + 2(2n) = p + 2k_1,
			\]
			para algum \(k_1 \in \mathbb{N}\).
			
			\item Para \(p + 4n + 2\), temos
			\[
			p + 4n + 2 = p + 2(2n + 1) = p + 2k_2,
			\]
			para algum \(k_2 \in \mathbb{N}\).
		\end{itemize}
		
		Como \(mdc(p, 2) = 1\), ambos os casos satisfazem as condições do  \textbf{Teorema de Dirichlet para progressões aritméticas}~\ref{teo:dirichlet}, que garante que toda progressão da forma \(a + kn\), com \(mdc(a, k) = 1\), contém infinitos números primos.
		
		Portanto, a sequência \(S_n(p)\) contém infinitos números primos.
	\end{proof}
	
	\begin{remark}
		Note que a definição ~\ref{def:snp} está relacionada diretamente ao Corolário ~\ref{coro:par_simetrico}.
	\end{remark}
	
	\begin{remark}
		Para um q = \(S_n(p)\): ou q é um primo ímpar ou q é um composto ímpar.
	\end{remark}
	
	\begin{proposition}
		Seja \( p > 2 \) um número primo e \( n \in \mathbb{N} \). Então, a soma \( S_n(p) + p \) é um número par.
	\end{proposition}
	
	
	\begin{proof}
		Por definição, \( S_n(p) \) é ímpar. De fato, como \( p > 2 \) é um número primo, ele também é ímpar. Assim, a soma de dois números ímpares é sempre um número par. Portanto, \( S_n(p) + p \) é par. 
	\end{proof}
	
	
	%%%%%%%%%%%%%%%%%%%%%%%%%%%%%%%%%%%%%%%
	\section{Resultados Principais}
	\begin{proposition}[Simetria dos Múltiplos]
		Seja \(a > 2\) um número par natural e seja \(p\) um número primo tal que \(p \mid a\). Então, o simétrico \(S_n(p)\) de \(p\) também é múltiplo de \(p\).
	\end{proposition}
	
	\begin{proof}
		Suponha, por absurdo, que o simétrico \(S_n(p) = q\) não seja múltiplo de \(p\). 
		
		Sendo \(a = q + p\) e assumindo que \(p \mid a\), segue que:
		
		\[
		p \mid (q + p)
		\]
		
		O que implica que:
		
		\[
		p \mid q
		\]
		
		O que é um absurdo, pois assumimos que \(q\) não é múltiplo de \(p\).
		
		Portanto, a suposição é falsa e conclui-se que \(S_n(p)\) é múltiplo de \(p\) sempre que \(p \mid a\).
	\end{proof}
	
	\begin{lemma}\label{lema:multip}
		Seja \(p > 2\) um número primo e \(q > p\) um número ímpar pertencente à sequência \(S_n(p)\). Se \(q\) é primo ou \(q\) não é múltiplo de \(p\), então
		
		\[
		2p \nmid (p + q)
		\]
	\end{lemma}
	\begin{proof}
		Dividimos a demonstração em dois casos:
		
		\textbf{Caso I:} \(q\) é múltiplo de \(p\).
		
		Se \(q\) é múltiplo de \(p\), então pela ProposiçãoX, \(S_n(p)\) também é multiplo de \(p\). Portanto,
		
		\[
		p \mid q \implies 2p \mid (p + q)
		\]
		Uma vez que \(p + q\) é par.
		Isso mostra que se \(q\) \textit{é múltiplo de} \(p\), então \(2p\) divide \(p + q\). O contrapositivo também é verdadeiro: se \(q\) \textit{não é múltiplo de} \(p\), então
		
		\[
		2p \nmid (p + q)
		\]
		
		\textbf{Caso II:} \(q\) é primo.
		
		Suponha, por absurdo, que \(2p \mid (p + q)\). Então, existe um inteiro \(k > 0\) tal que
		
		\[
		p + q = 2p k
		\]
		
		Isolando \(q\), obtemos
		
		\[
		q = 2p k - p = p(2k - 1)
		\]
		
		Ou seja, \(q\) é múltiplo de \(p\). Como \(p\) é um primo estritamente menor que \(q\) (pois \(q > p\)), isso implica que \(q\) é um número composto, o que contradiz a hipótese de que \(q\) é primo.
		
		Portanto, se \(q\) é primo,
		
		\[
		2p \nmid (p + q)
		\]
		
		o que conclui a demonstração.
	\end{proof}
	
	\begin{corollary}
		Seja \(p > 2\) um número primo e \(n \in \mathbb{N}\). Então, todo número par da forma \(p + S_n(p)\), tal que \(S_n(p) > p\), não é múltiplo de \(p\). Ou seja,
		
		\[
		p \nmid (p + S_n(p)) \quad \text{sempre que} \quad S_n(p) > p
		\]
		
	\end{corollary}
	
	\begin{proof}
		Segue imediatamente do Lema ~\ref{lema:multip}, observando que a única possibilidade de múltiplo de \(p\) ocorre quando \(S_n(p) = p\), o que é excluído pela condição \(S_n(p) > p\).
	\end{proof}
	
	
	\begin{proposition}
		Seja \(a \in \mathbb{N}\) um número par tal que \(a > 6\). Então, não existe nenhum conjunto de primos \(P = \{p \in \mathbb{P} : p < \frac{a}{2}\}\) tal que
		
		\[
		a = \prod_{p < \frac{a}{2}} p
		\]
		
		Em outras palavras, nenhum número par \(a > 6\) pode ser expresso como o produto de todos os primos estritamente menores que \(\frac{a}{2}\).
	\end{proposition}
	
	\begin{proof}
		Seja \(a\) um número par tal que \(a > 6\). Seja
		
		\[
		P(a) = \prod_{p < \frac{a}{2}} p
		\]
		
		o produto de todos os primos estritamente menores que \(\frac{a}{2}\).
		
		Observamos que para \(a > 6\) vale \(\frac{a}{2} > 3\), portanto o conjunto de primos menores que \(\frac{a}{2}\) contém pelo menos os primos 2 e 3, e frequentemente mais.
		
		O produto \(P(a)\) cresce de forma supermultiplicativa, já que cada primo adicionado ao produto faz com que seu valor cresça pelo menos multiplicando pelo menor primo seguinte.
		
		Basta verificar que para qualquer \(a > 6\), o produto \(P(a)\) é estritamente maior que \(a\):
		
		- Se \(\frac{a}{2} \geq 5\), então \(P(a) \geq 2 \cdot 3 \cdot 5 = 30\), e como \(\frac{a}{2} \geq 5\), então \(a \geq 10\), mas \(30 > 10\), logo a igualdade é impossível.
		
		- Para valores maiores de \(a\), o número de primos no produto só aumenta, e consequentemente \(P(a)\) cresce rapidamente.
		
		Logo, não existe \(a > 6\) par tal que
		
		\[
		a = \prod_{p < \frac{a}{2}} p
		\]
		
	\end{proof}
	
	\begin{proposition}[Existência de Primos Não Divisores]
		Seja \(a \in \mathbb{N}\) um número par tal que \(a > 6\). Então, existe pelo menos um primo \(p \in \mathbb{P}\), com \(p < \frac{a}{2}\), tal que \(p\) não divide \(a\).
	\end{proposition}
	
	\begin{proof}
		Por contradição, suponha que todo primo \(p < \frac{a}{2}\) divide \(a\). Isso implicaria que \(a\) é múltiplo do produto de todos os primos estritamente menores que \(\frac{a}{2}\), ou seja,
		
		\[
		a = k \cdot \prod_{p < \frac{a}{2}} p
		\]
		
		para algum \(k \in \mathbb{N}\). No entanto, isso contradiz diretamente a Proposição~\ref{lema_produto_primos}, que estabelece que nenhum número par \(a > 6\) pode ser expresso como o produto de todos os primos menores que \(\frac{a}{2}\).
		
		Portanto, nossa suposição leva a um absurdo. Assim, necessariamente, existe pelo menos um primo \(p < \frac{a}{2}\) tal que \(p\) não divide \(a\).
		
		Isso conclui a demonstração.
	\end{proof}
	
	\subsection{Proposta de Demonstração para a conjectura de Goldbach}
	\begin{definition}[Conjunto \(Q_p\)]
		Seja \(p \in P\) um número primo fixo com \(p > 2\). Definimos \(Q_p\) como o subconjunto dos números primos pertencentes à sequência \(S_n(p)\), ou seja:
		
		\[
		Q_p = \left\{ q \in S_n(p) \mid q \text{ é primo} \right\}
		\]
		
		onde \(S_n(p)\) é a sequência dos números ímpares construídos com base na simetria dos números não múltiplos de \(p\), conforme definido anteriormente. O conjunto \(Q_p\) representa, portanto, todos os primos que pertencem à sequência \(S_n(p)\) associada a cada primo \(p\).
	\end{definition}
	
	\begin{definition}[Conjunto \(A_p\)]
		Para um primo \(p \in P\) com \(p > 2\), definimos:
		
		\[
		A_p = \{p + q \mid q \in Q_p\}
		\]
		
		onde \(Q_p\) é o subconjunto dos números primos pertencentes à sequência \(S_n(p)\). O conjunto \(A_p\) contém todos os pares gerados pela soma do primo \(p\) com os primos de sua sequência associada.
	\end{definition}
	
	\begin{lemma}[Cobertura dos Números Pares pelos Subconjuntos \(A_i\)]
		Seja \(m\) um número par tal que \(m > 2\). Então, existe pelo menos um primo \(p \in P\) tal que \(p\) não divide \(m\), e, portanto, \(m\) pertence ao subconjunto \(A_p\), onde:
		
		\[
		A_p = \{p + q \mid q \in Q_p\}
		\]
		
		com \(Q_p\) sendo o conjunto dos primos pertencentes à sequência \(S_n(p)\) associada a \(p\).
	\end{lemma}
	
	\begin{proof}
		Suponha, por absurdo, que existe um número par \(m > 2\) tal que \(m\) não pertence a nenhum subconjunto \(A_p\) para todo \(p \in P\). Isso implica que, para todo primo \(p\), o número \(m\) é múltiplo de \(p\).
		
		Contudo, não existe número natural finito que seja múltiplo de todos os primos, pois o produto de infinitos primos diverge para o infinito. Portanto, é impossível que \(m\) seja múltiplo de todos os primos do conjunto \(P\).[Proposicao 3.3 e 3.4]
		
		Logo, necessariamente, existe pelo menos um primo \(p\) tal que \(p\) não divide \(m\). Consequentemente, como as sequências \(S_n(p)\) são construídas excluindo os múltiplos de \(p\), o número \(m\) pertence ao subconjunto \(A_p = \{p + q \mid q \in Q_p\}\), onde \(q\) percorre os primos da sequência \(S_n(p)\).
		
		Isso conclui a demonstração.
	\end{proof}
	\begin{remark}
		Este lema reforça que, embora cada subconjunto \(A_p\) exclua especificamente os múltiplos de \(p\), a união dos subconjuntos cobre integralmente o conjunto dos números pares maiores que dois. Isso decorre do fato de que não existe número par finito que seja múltiplo de todos os primos.
	\end{remark}
	
	
	\begin{theorem}[Cobertura dos Números Pares]\label{teo:principal}
		Seja 
		\[
		B = \left\{ (a_n)_{n \in \mathbb{N}} \mid a_n = p + q, \, p \in P, \, q \in Q_p \right\}
		\]
		onde \(P\) é o conjunto dos números primos maiores que três, e \(Q_p\) é um subconjunto dos primos pertencentes à sequência \(S_n(p)\). 
		
		Então,
		\[
		\bigcup_{n \in \mathbb{N}} a_n = A
		\]
		onde \(A\) é o conjunto dos números naturais pares maiores que dois.
		
		Em outras palavras, todo número par maior que dois pode ser expresso como a soma de um primo \(p \in P\) e um primo \(q \in Q_p\).
	\end{theorem}
	
	\begin{proof}
		Seja \(p \in P\) com \(p > 2\), e seja \(Q_p\) o subconjunto dos números primos pertencentes à sequência \(S_n(p)\), conforme definido na Definição 2.5. Pelo Lema 3.1 e pelo Corolário 3.2, sabemos que a sequência \(p + S_n(p)\) com \(p > 2\) gera todos os números pares que não são múltiplos de \(p\), com exceção do termo específico \(2p\), que é o único múltiplo de \(p\) presente. Ou seja, \(p + S_n(p)\) gera todos o pares maiores que \(2p\) que não são múltiplos de \(p\).
		
		Definimos, então, para cada \(i \in \mathbb{N}\) associado a um primo \(p_i \in P\), o conjunto:
		
		\[
		A_i = \{p_i + q_j \mid q_j \in Q_{p_i}\}
		\]
		
		onde cada termo \(p_i + q_j\) é a soma do primo fixo \(p_i\) com um primo \(q_j\) pertencente à sequência \(S_n(p_i)\).
		
		Observamos que, com exceção de \(2p_i\), todos os elementos de \(A_i\) não são múltiplos de \(p_i\), dado que os termos da sequência \(S_n(p_i)\) são construídos justamente para evitar os múltiplos de \(p_i\), salvo o termo inicial \(2p_i\) quando presente. Além disso, aqueles que não são múltiplos de \(p_i\) podem ser múltiplos de primos menores (se houver) ou maiores, mas não de \(p_i\) especificamente.
		
		Dessa forma, os conjuntos \(A_i\) preenchem, de forma distribuída, os pares não múltiplos dos respectivos primos \(p_i\), e a sobreposição entre esses conjuntos é evitada, salvo nos casos em que um número é múltiplo de outros primos diferentes de \(p_i\).
		
		Assim, a união dos conjuntos \(A_i\), ou seja,
		
		\[
		\bigcup_{i \in \mathbb{N}} A_i
		\]
		
		cobre integralmente o conjunto dos números pares maiores que dois, sem omitir elementos e sem redundância sistemática de termos, exceto por eventuais múltiplos comuns a diferentes primos.
		
		Nota-se, adicionalmente, que estamos considerando apenas \(p > 2\). O caso \(p = 2\) é especial, pois a sequência \(S_n(2)\) se reduz a duas progressões específicas:
		
		\[
		S_n(2) =
		\begin{cases}
			4n + 2 \\
			4n + 4
		\end{cases}
		\]
		
		que não satisfazem as condições do Teorema de Dirichlet, pois \(mdc(4,2) \neq 1\) e \(mdc(4,4) \neq 1\). Portanto, \(p = 2\) não é incluído na definição geral do teorema.
		
		Se estendermos a definição formalmente para \(p = 2\), o único termo válido seria quando \(n = 0\), ou seja,
		
		\[
		a_1 = \{4\}
		\]
		
		Por fim, considerando qualquer \(a_n \in B\), temos que todos os números pares da união das sequências \(A_i\) pertencem a \(A\), isto é,
		
		\[
		\bigcup_{n \in \mathbb{N}} a_n = A
		\]
		
		Encerrando, portanto, a demonstração do teorema.
	\end{proof}
	\begin{corollary}[Par Simétrico de Primos]\label{coro:par_simetrico_primos}
		Todo número par \(a > 2\) possui um par simétrico de primos \(p\) e \(q \in S_n(p)\) tal que:
		
		\[
		a = p + S_n(p)
		\]
	\end{corollary}
	
	\begin{proof}
		Seja \(a\) um número par tal que \(a > 2\). Pelo Teorema 3.3, sabemos que:
		
		\[
		a \in \bigcup_{i \in \mathbb{N}} A_i
		\]
		
		onde cada subconjunto \(A_i\) é dado por:
		
		\[
		A_i = \{p_i + q \mid q \in S_n(p_i)\}
		\]
		
		com \(p_i\) um primo pertencente a \(P\) e \(q\) um primo pertencente à sequência \(S_n(p_i)\).
		
		Por definição, todo elemento \(a \in A_i\) pode ser escrito como:
		
		\[
		a = p_i + S_n(p_i)
		\]
		
		com \(p_i\) e \(S_n(p_i)\) ambos números primos. Como a sequência \(S_n(p_i)\) é simétrica em relação ao primo \(p_i\), de acordo com o Corolário~\ref{coro:par_simetrico} e com a Definição~\ref{def:snp}, então qualquer número par \(a \in A_i\) possui um par simétrico de primos.
		
		Logo, todo número par \(a > 2\) pode ser representado como:
		
		\[
		a = p + q
		\]
		
		com \(p\) e \(q\) sendo primos, onde \(q \in S_n(p)\).
		
		Esse resultado é garantido diretamente pelo Teorema 3.3, uma vez que:
		
		\[
		\bigcup_{i \in \mathbb{N}} A_i = A
		\]
		
		o que completa a demonstração. 
	\end{proof}
	
	
	\begin{theorem}[Formulação Simétrica da Conjectura de Goldbach]\label{teo:goldbach_por_simetria_de_primos}
		Todo número par \(a > 2\) pode ser expresso como a soma de dois números primos \(p\) e \(q\), tais que:
		
		\[
		a = p + q
		\]
		
		onde \(p\) é um número primo pertencente ao conjunto \(P\) (primos maiores que dois) e \(q\) é um número primo pertencente à sequência \(S_n(p)\) associada a \(p\), conforme definida neste trabalho.
		
		Em outras palavras, para todo número par \(a > 2\), existe um par simétrico de primos \((p, q)\), sendo \(q \in S_n(p)\), tal que:
		
		\[
		a = p + S_n(p)
		\]
	\end{theorem}
	\begin{proof}
	O resultado segue diretamente do Teorema~\ref{teo:principal} (Cobertura dos Números Pares) e do Corolário~\ref{coro:par_simetrico_primos}, os quais estabelecem que todo número par maior que dois pertence à união dos subconjuntos formados por somas da forma \(p + q\), com \(p\) pertencente ao conjunto dos primos \(P\) e \(q\) pertencente à sequência \(S_n(p)\). Como demonstrado, a sequência \(S_n(p)\) contém infinitos primos e possui relação de simétrica com \(p\), garantindo que todo número par pode ser expresso como a soma de um primo \(p\) e de um primo \(q \in S_n(p)\).
	
	Ressalta-se que o caso \(a = 4\) é gerado diretamente como \(4 = 2 + 2\), através da sequência \(S_n(2)\), que constitui um caso particular dentro da construção. Para todos os outros primos \(p > 2\), as propriedades estruturais da sequência \(S_n(p)\) são plenamente aplicáveis conforme as definições apresentadas.
	
	Isso conclui a demonstração.
	\end{proof}
	
	
	
	
	\section{Discussão do Corolário~\ref{coro:par_simetrico_primos} e do Teorema~\ref{teo:goldbach_por_simetria_de_primos} e Implicações na Conjectura de Goldbach}
	
	O Corolário~\ref{coro:par_simetrico_primos} estabelecido no contexto deste trabalho demonstra que todo número par \(a > 2\) possui um \textbf{par simétrico de primos}, especificamente da forma \(a = p + q\), onde \(p\) é um primo fixado pertencente ao conjunto \(P\) (primos maiores que três) e \(q\) é um primo pertencente à sequência \(S_n(p)\). O Teorema~\ref{teo:goldbach_por_simetria_de_primos} estende esse resultado a Conjectura de Goldbach a fim de demonstrá-lo. 
	
	\subsection{Interpretação Estrutural}
	A sequência \(S_n(p)\) possui uma construção baseada na simetria dos números ímpares menores que um determinado valor, conforme definido anteriormente na Proposição~\ref{prop:distancia_simetrica}, no Corolário~\ref{coro:par_simetrico} e na Definiçãp~\ref{def:snp}. Dessa forma, para cada primo \(p\), a sequência \(S_n(p)\) fornece um subconjunto de primos que, somados a \(p\), geram números pares que não são múltiplos de  \(p\), exceto pelo termo  \(2p\), mas, coletivamente, a união desses pares cobre \textbf{todo o conjunto dos pares maiores que dois}, como garantido pelo Teorema~\ref{teo:principal}.
	
	\subsection{Implicações Diretas na Conjectura de Goldbach}
	Este resultado sustenta, sob a construção proposta, que qualquer número par maior que dois \textbf{pode ser expresso como a soma de dois primos}, sendo um deles um primo fixo \(p\) e o outro um primo proveniente da sequência \(S_n(p)\) associada a esse \(p\).
	
	A simetria observada na construção da sequência \(S_n(p)\) não apenas sugere, mas também estrutura formalmente a geração dos pares como combinações específicas de primos e seus simétricos dentro das sequências.
	
	\subsection{Observação de Relevância}
	Ainda que o Teorema~\ref{teo:goldbach_por_simetria_de_primos} possa não constituir uma demonstração completa da Conjectura de Goldbach no sentido clássico, este trabalho fornece uma estrutura formal elaborada e uma decomposição do conjunto dos pares em subconjuntos que, isoladamente, são gerados pela soma de primos específicos com seus simétricos nas sequências \(S_n(p)\).
	
	Essa abordagem pode, portanto, representar uma estratégia alternativa na busca de uma demonstração completa, ao invés de se focar exclusivamente na busca de por um par de primos qualquer, opta-se pela decomposição sistemática do conjunto dos pares, distribuindo-os segundo as progressões e simetrias associadas a cada primo \(p\).
	
	\subsection{Perspectivas}
	Essa formulação permite avançar na análise estrutural do problema, podendo conduzir a novos lemas e teoremas que, eventualmente, possam atacar os casos residuais ou demonstrar a exaustividade da cobertura para todos os números pares.
	
	Além disso, a abordagem sugere a possibilidade de análise algorítmica, uma vez que cada subconjunto \(A_i\) pode ser gerado computacionalmente a partir de um primo \(p_i\) e dos elementos da sequência \(S_n(p_i)\).
	
	\subsection{Conclusão da Discussão}
	O Corolário~\ref{coro:par_simetrico_primos}, portanto, não apenas reforça a validade da construção proposta, como também fornece uma formulação que conecta diretamente a Conjectura de Goldbach com o Teorema de Dirichlet aplicado às progressões associadas às sequências \(S_n(p)\), e com as propriedades de simetria dos números ímpares menores que um determinado limite. Além disso, esse resultado é a principal para demonstrar o Teorema~\ref{teo:goldbach_por_simetria_de_primos} deste trabalho.
	
	
	%%%%%%%%%%%%%%%%%%%%%%%%%%%%%%%%%%%%%%%
	\section{Conclusão}
	
	Este trabalho apresentou uma abordagem alternativa para a Conjectura de Goldbach, fundamentada na decomposição dos números pares maiores que dois em somas da forma \(p + q\), onde \(p\) é um primo fixo e \(q\) pertence a uma sequência específica \(S_n(p)\) associada a esse primo. A partir da aplicação do Teorema de Dirichlet e das propriedades de simetria dos números ímpares, foi possível demonstrar que a união dos subconjuntos formados por essas somas gera integralmente o conjunto dos números pares maiores que dois.
	
	Embora haja a possibilidade do resultado obtido não configurar uma prova definitiva da Conjectura de Goldbach através do Teorema~\ref{teo:goldbach_por_simetria_de_primos}, ele oferece uma contribuição significativa no sentido de estruturar o problema sob uma nova ótica, organizando os números pares segundo padrões algébricos e simétricos. Além disso, essa abordagem abre espaço para a formulação de novos lemas, conjecturas auxiliares e estratégias de demonstração que podem ser exploradas em pesquisas futuras.
	
	Como trabalhos futuros, sugere-se aprofundar a análise das interseções entre os subconjuntos gerados, investigar a distribuição dos primos dentro das sequências \(S_n(p)\) e explorar a possibilidade de estender essa abordagem para outros problemas clássicos da Teoria dos Números. A proposta aqui apresentada também pode ser adaptada para análises computacionais, permitindo a geração e a verificação automatizada das estruturas formadas, contribuindo assim tanto para a pesquisa teórica quanto para o desenvolvimento de algoritmos relacionados à distribuição dos números primos.
	
	%%%%%%%%%%%%%%%%%%%%%%%%%%%%%%%%%%%%%%%
	\section*{Agradecimentos}
	
	Ofereço este trabalho em homenagem a todos os meus professores de matemática, que contribuíram para minha formação e paixão pela matemática.
	
	%%%%%%%%%%%%%%%%%%%%%%%%%%%%%%%%%%%%%%%
	\begin{thebibliography}{9}
		
		\bibitem{apostol}
		T. M. Apostol.
		\textit{Introduction to Analytic Number Theory}.
		Springer, 1976.
		
		\bibitem{paiva}
		C. D. C. Paiva, et al. \textit{A busca pela prova da conjectura de Goldbach: explorando suas conquistas}. Caderno Pedagógico, 21(7): e5770-e5770, 2024.
		
		
		
		
	\end{thebibliography}
	%\bibliographystyle{amsplain} 
	%\bibliography{bio} 
	
\end{document}
