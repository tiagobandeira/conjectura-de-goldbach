\documentclass[a4paper,11pt]{article}

% Pacotes básicos
\usepackage[utf8]{inputenc} % Codificação
\usepackage[T1]{fontenc}    % Fontes com acentuação
\usepackage[brazil]{babel}  % Idioma
\usepackage{amsmath, amssymb, amsthm} % Matemática avançada
\usepackage{geometry}       % Margens
\usepackage{graphicx}       % Imagens
\usepackage{hyperref}       % Links
\usepackage{lmodern}        % Fonte melhorada
\usepackage{mathrsfs}       % Fontes matemáticas adicionais
\usepackage{enumitem}       % Listas personalizadas

% Configurações de página
\geometry{
	a4paper,
	left=30mm,
	right=30mm,
	top=25mm,
	bottom=25mm,
}

% Comandos para título
\title{\textbf{Uma Proposta de Demonstração para a Conjectura de Goldbach Utilizando o Teorema de Dirichlet}\\
	\large (Preprint)}
\author{
	Tiago Bandeira\thanks{tiagobandeirabarros@gmail.com} 
}
\date{\today}

% Teoremas e definições
\newtheorem{theorem}{Teorema}[section]
\newtheorem{lemma}[theorem]{Lema}
\newtheorem{proposition}[theorem]{Proposição}
\newtheorem{corollary}[theorem]{Corolário}
\theoremstyle{definition}
\newtheorem{definition}[theorem]{Definição}
\theoremstyle{remark}
\newtheorem{remark}[theorem]{Observação}
\newtheorem{example}[theorem]{Exemplo}

% Início do documento
\begin{document}
	
	\maketitle
	
	\begin{abstract}
		Este trabalho visa abordar a Conjectura de Goldbach e utiliza o Teorema de Dirichlet sobre progressões aritméticas para tentar demonstrar o resultado principal presente neste artigo. Tal resultado mostra que para todo primo p > 2 e os q's da sequência S{n}(P) que são primos, a soma p + q é capaz de gerar todos os números pares naturais que são maiores que 2. Ou seja, é possível construir o conjunto de todos os pares maiores que 2 por meio da união dos subconjuntos gerados por P + S{n}(P). Por mais que esse resultado não seja uma prova definitiva e talvez não inovadora, essa ideia pode contribuir sistematicamente como ponto de partida para uma forma de demonstração em trabalhos futuros. A sequência  S{n}(P) bem como as demais ideias ditas nesse resumo serão apresentadas no decorrer do artigo.
	\end{abstract}
	
	\textbf{Palavras-chave:} conjectura de goldbach; números primos; teoria dos números.
	
	%\vspace{0.5cm}
	% sumario
	%\tableofcontents
	\vspace{1cm}
	
	%%%%%%%%%%%%%%%%%%%%%%%%%%%%%%%%%%%%%%%
	\section{Introdução}
	
	A introdução apresenta o contexto do problema, motivações, trabalhos relacionados e a organização do artigo.
	
	%%%%%%%%%%%%%%%%%%%%%%%%%%%%%%%%%%%%%%%
	\section{Fundamentação Teórica}
	
	Aqui você descreve os conceitos, definições e ferramentas matemáticas utilizadas no desenvolvimento do trabalho.
	\begin{proposition}
		Seja $a \in \mathbb{N}$, com $a > 2$ e $a$ par. Considere a sequência ordenada dos números ímpares estritamente menores que $a$:
		
		\[
		S = \{1, 3, 5, \dotsc, a - 1\}
		\]
		
		Então, para todo $k \in S$, vale que:
		
		\[
		a = k + (a - k)
		\]
		
		e ambos $k$ e $a - k$ pertencem a $S$, sendo simétricos em relação aos termos centrais da sequência.
	\end{proposition}
	
	
	\begin{proof}
		Observemos que os elementos de $S$ podem ser descritos pela progressão aritmética:
		
		\[
		S = \{2n - 1 \mid n = 1, 2, \dotsc, \tfrac{a}{2} \}
		\]
		
		O número total de elementos da sequência é $\tfrac{a}{2}$, inteiro pois $a$ é par.
		
		Seja um elemento arbitrário $k = 2n - 1$, com $n \in \{1, 2, \dotsc, \tfrac{a}{2}\}$. Calculamos seu par simétrico:
		
		\[
		a - k = a - (2n - 1) = (a - 2n) + 1
		\]
		
		Note que $(a - 2n)$ é par, e portanto $(a - 2n) + 1$ é ímpar. Além disso, verificamos que:
		
		\[
		a - k \leq a - 1
		\]
		\[
		a - k \geq 1
		\]
		
		Logo, $a - k \in S$.
		
		A soma do par $(k, a - k)$ resulta em:
		
		\[
		k + (a - k) = (2n - 1) + (a - (2n - 1)) = a
		\]
		
		Portanto, qualquer elemento de $S$ somado ao seu simétrico gera exatamente $a$.
		
	\end{proof}
	
	\subsection*{Exemplos}
	
	Seja $a = 12$:
	
	\[
	S = \{1, 3, 5, 7, 9, 11\}
	\]
	Os pares simétricos são:
	\[
	1 + 11 = 12, \quad 3 + 9 = 12, \quad 5 + 7 = 12
	\]
	
	Seja $a = 10$:
	
	\[
	S = \{1, 3, 5, 7, 9\}
	\]
	Os pares simétricos são:
	\[
	1 + 9 = 10, \quad 3 + 7 = 10, \quad 5 + 5 = 10
	\]
	O termo central $5$ se emparelha consigo mesmo.
	
	\subsection*{Conclusão}
	
	Dessa forma, qualquer número par $a > 2$ pode ser decomposto como a soma de dois números ímpares menores que $a$, escolhidos de forma simétrica em relação aos termos centrais da sequência dos ímpares menores que $a$.
		
	\vspace{1cm}
	
	\begin{proposition}
		Seja $a \in \mathbb{N}$, com $a > 2$ e $a$ par, e seja
		\[
		S = \{1, 3, 5, \dotsc, a - 1\}
		\]
		a sequência dos números ímpares menores que $a$.
		
		Sejam $b, c \in S$ dois termos simétricos em relação aos termos centrais da sequência, e seja $i$ a distância posicional de $b$ até o centro da sequência $S$. Então, a diferença entre $c$ e $b$ satisfaz:
		\[
		|c - b| =
		\begin{cases}
			4i, & \text{se } |S| \text{ é ímpar} \\
			4i + 2, & \text{se } |S| \text{ é par}
		\end{cases}
		\]
		onde $|S| = \frac{a}{2}$ é o número de elementos da sequência.
	\end{proposition}
	
	\begin{proof}
		A sequência $S$ possui $N = \frac{a}{2}$ termos, com
		\[
		S = \{2n - 1 \mid n = 1, 2, \dotsc, N\}.
		\]
		
		Considere $b$ o termo na posição $n$, com $1 \leq n \leq N$, e seu termo simétrico $c$ na posição $N - n + 1$. Assim,
		\[
		b = 2n - 1, \quad c = 2(N - n + 1) - 1 = 2N - 2n + 1.
		\]
		
		A diferença entre os termos é:
		\[
		|c - b| = |(2N - 2n + 1) - (2n - 1)| = |2N - 4n + 2| = 2|N - 2n + 1|.
		\]
		
		Definimos $i$ como a distância posicional de $b$ até o centro da sequência. Dependendo da paridade de $N$, o centro é definido como:
		
		\begin{itemize}
			\item Se $N$ é ímpar, existe um termo central único na posição $m = \frac{N+1}{2}$. Assim,
			\[
			i = |n - m| = \left| n - \frac{N+1}{2} \right|.
			\]
			Neste caso,
			\[
			|N - 2n + 1| = 2i,
			\]
			portanto,
			\[
			|c - b| = 2 \times 2i = 4i.
			\]
			
			\item Se $N$ é par, existem dois termos centrais nas posições $m_1 = \frac{N}{2}$ e $m_2 = \frac{N}{2} + 1$. Para $b$ à esquerda do centro,
			\[
			i = m_1 - n \implies 2i = |N - 2n|,
			\]
			e
			\[
			|N - 2n + 1| = 2i + 1,
			\]
			portanto,
			\[
			|c - b| = 2(2i + 1) = 4i + 2.
			\]
		\end{itemize}
		
		Assim, mostramos que a diferença entre termos simétricos na sequência $S$ segue a fórmula desejada, dependendo da paridade de $N$.
		
	\end{proof}
	
	\begin{corollary} \label{coro:par_simetrico}
		Sejam $a \in \mathbb{N}$, com $a > 2$ e $a$ par, e
		\[
		S = \{1, 3, 5, \dotsc, a - 1\}
		\]
		a sequência dos números ímpares menores que $a$.
		
		Seja $b \in S$ um termo a uma distância posicional $i$ do centro da sequência. Então, seu termo simétrico $c \in S$ satisfaz:
		
		\[
		c =
		\begin{cases}
			b + 4i, & \text{se } |S| \text{ é ímpar} \\
			b + 4i + 2, & \text{se } |S| \text{ é par}
		\end{cases}
		\]
		onde $|S| = \frac{a}{2}$ e $i$ é a distância posicional de $b$ até o centro da sequência.
	\end{corollary}
	
	
	
	\begin{theorem}[Teorema de Dirichlet para Progressões Aritméticas] \label{teo:dirichlet}
		Sejam $a$ e $d$ inteiros positivos tais que $mdc(a, d) = 1$. Então, a progressão aritmética da forma
		
		\[
		a,\, a + d,\, a + 2d,\, a + 3d,\, \dots
		\]
		
		contém infinitos números primos.
	\end{theorem}
	
	O Teorema de Dirichlet sobre progressões aritméticas garante que toda progressão da forma $a + nd$, com $mdc(a, d) = 1$, contém infinitos números primos.~Por ser complexa e envolver ferramentas matemáticas além do escopo deste trabalho a prova do Teorema de Dirichlet não será demonstrada aqui. Porém é possível encontrá-la em ~\cite{apostol}.  
	
	\begin{definition}[Sequência \(S_n(p)\)]\label{def:snp}
		Seja \(p > 2\) um número primo fixado. Definimos a sequência \(S_n(p)\), com \(n \in \mathbb{N}\), como o conjunto dos números ímpares da forma
		\[
		S_n(p) = 
		\begin{cases}
			p + 4n \\
			p + 4n + 2
		\end{cases}
		\]
		para todo \(n \in \mathbb{N}\).
	\end{definition}
	
	\vspace{0.5cm}
	
	\begin{proposition}
		A sequência \(S_n(p)\) contém infinitos números primos.
	\end{proposition}
	
	\begin{proof}
		Observamos que, nos dois casos,
		\begin{itemize}
			\item Para \(p + 4n\), temos
			\[
			p + 4n = p + 2(2n) = p + 2k_1,
			\]
			para algum \(k_1 \in \mathbb{N}\).
			
			\item Para \(p + 4n + 2\), temos
			\[
			p + 4n + 2 = p + 2(2n + 1) = p + 2k_2,
			\]
			para algum \(k_2 \in \mathbb{N}\).
		\end{itemize}
		
		Como \(mdc(p, 2) = 1\), ambos os casos satisfazem as condições do  \textbf{Teorema de Dirichlet para progressões aritméticas}~\ref{teo:dirichlet}, que garante que toda progressão da forma \(a + kn\), com \(mdc(a, k) = 1\), contém infinitos números primos.
		
		Portanto, a sequência \(S_n(p)\) contém infinitos números primos.
	\end{proof}
	
	\begin{remark}
		Note que a definição ~\ref{def:snp} está relacionada diretamente ao Corolário ~\ref{coro:par_simetrico}.
	\end{remark}
	
	\begin{remark}
		Para um q = \(S_n(p)\): ou q é um primo ímpar ou q é um composto ímpar.
	\end{remark}
	
	\begin{proposition}
		Seja \( p > 2 \) um número primo e \( n \in \mathbb{N} \). Então, a soma \( S_n(p) + p \) é um número par.
	\end{proposition}
	
	
	\begin{proof}
		Por definição, \( S_n(p) \) é ímpar. De fato, como \( p > 2 \) é um número primo, ele também é ímpar. Assim, a soma de dois números ímpares é sempre um número par. Portanto, \( S_n(p) + p \) é par. 
	\end{proof}
	
	
	
	
	%%%%%%%%%%%%%%%%%%%%%%%%%%%%%%%%%%%%%%%
	\section{Resultados Principais}
	\begin{lemma}\label{lema:multip}
		Seja \(p > 2\) um número primo e \(q > p\) um número ímpar pertencente à sequência \(S_n(p)\). Se \(q\) é primo ou \(q\) não é múltiplo de \(p\), então
		
		\[
		2p \nmid (p + q)
		\]
	\end{lemma}
	\begin{proof}
		Dividimos a demonstração em dois casos:
		
		\textbf{Caso I:} \(q\) é múltiplo de \(p\).
		
		Se \(q\) é múltiplo de \(p\), então existe um inteiro \(k\) tal que
		
		\[
		q = pk
		\]
		
		Logo,
		
		\[
		p + q = p + pk = p(1 + k)
		\]
		
		Portanto,
		
		\[
		2p \mid p(1 + k) \implies 2p \mid (p + q)
		\]
		
		Isso mostra que se \(q\) \textit{é múltiplo de} \(p\), então \(2p\) divide \(p + q\). O contrapositivo também é verdadeiro: se \(q\) \textit{não é múltiplo de} \(p\), então
		
		\[
		2p \nmid (p + q)
		\]
		
		\textbf{Caso II:} \(q\) é primo.
		
		Suponha, por absurdo, que \(2p \mid (p + q)\). Então, existe um inteiro \(k > 0\) tal que
		
		\[
		p + q = 2p k
		\]
		
		Isolando \(q\), obtemos
		
		\[
		q = 2p k - p = p(2k - 1)
		\]
		
		Ou seja, \(q\) é múltiplo de \(p\). Como \(p\) é um primo estritamente menor que \(q\) (pois \(q > p\)), isso implica que \(q\) é um número composto, o que contradiz a hipótese de que \(q\) é primo.
		
		Portanto, se \(q\) é primo,
		
		\[
		2p \nmid (p + q)
		\]
		
		o que conclui a demonstração.
	\end{proof}
	
	\begin{corollary}
		Seja \(p > 2\) um número primo e \(n \in \mathbb{N}\). Então, todo número par da forma \(p + S_n(p)\), tal que \(S_n(p) > p\), não é múltiplo de \(p\). Ou seja,
		
		\[
		p \nmid (p + S_n(p)) \quad \text{sempre que} \quad S_n(p) > p
		\]
		
	\end{corollary}
	
	\begin{proof}
		Segue imediatamente do Lema ~\ref{lema:multip}, observando que a única possibilidade de múltiplo de \(p\) ocorre quando \(S_n(p) = p\), o que é excluído pela condição \(S_n(p) > p\).
	\end{proof}
	\subsection{Proposta de Demonstração para a conjectura de Goldbach}
	\begin{theorem}[Cobertura dos Números Pares]
		Seja 
		\[
		B = \left\{ (a_n)_{n \in \mathbb{N}} \mid a_n = p + q, \, p \in P, \, q \in Q_p \right\}
		\]
		onde \(P\) é o conjunto dos números primos maiores que três, e \(Q_p\) é um subconjunto dos primos pertencentes à sequência \(S_n(p)\). 
		
		Então,
		\[
		\bigcup_{n \in \mathbb{N}} a_n = A
		\]
		onde \(A\) é o conjunto dos números naturais pares maiores que dois.
		
		Em outras palavras, todo número par maior que dois pode ser expresso como a soma de um primo \(p \in P\) e um primo \(q \in Q_p\).
	\end{theorem}
	
	\begin{proof}
		Seja \(p \in P\) com \(p > 2\), e seja \(Q_p\) o subconjunto dos números primos pertencentes à sequência \(S_n(p)\), conforme definido na Definição 2.5. Pelo Lema 3.1 e pelo Corolário 3.2, sabemos que a sequência \(p + S_n(p)\) com \(p > 2\) gera todos os números pares que não são múltiplos de \(p\), com exceção do termo específico \(2p\), que é o único múltiplo de \(p\) presente.
		
		Definimos, então, para cada \(i \in \mathbb{N}\) associado a um primo \(p_i \in P\), o conjunto:
		
		\[
		A_i = \{p_i + q_j \mid q_j \in Q_{p_i}\}
		\]
		
		onde cada termo \(p_i + q_j\) é a soma do primo fixo \(p_i\) com um primo \(q_j\) pertencente à sequência \(S_n(p_i)\).
		
		Observamos que, com exceção de \(2p_i\), todos os elementos de \(A_i\) não são múltiplos de \(p_i\), dado que os termos da sequência \(S_n(p_i)\) são construídos justamente para evitar os múltiplos de \(p_i\), salvo o termo inicial \(2p_i\) quando presente. Além disso, aqueles que não são múltiplos de \(p_i\) podem ser múltiplos de primos menores (se houver) ou maiores, mas não de \(p_i\) especificamente.
		
		Dessa forma, os conjuntos \(A_i\) preenchem, de forma distribuída, os pares não múltiplos dos respectivos primos \(p_i\), e a sobreposição entre esses conjuntos é evitada, salvo nos casos em que um número é múltiplo de outros primos diferentes de \(p_i\).
		
		Assim, a união dos conjuntos \(A_i\), ou seja,
		
		\[
		\bigcup_{i \in \mathbb{N}} A_i
		\]
		
		cobre integralmente o conjunto dos números pares maiores que dois, sem omitir elementos e sem redundância sistemática de termos, exceto por eventuais múltiplos comuns a diferentes primos.
		
		Nota-se, adicionalmente, que estamos considerando apenas \(p > 2\). O caso \(p = 2\) é especial, pois a sequência \(S_n(2)\) se reduz a duas progressões específicas:
		
		\[
		S_n(2) =
		\begin{cases}
			4n + 2 \\
			4n + 4
		\end{cases}
		\]
		
		que não satisfazem as condições do Teorema de Dirichlet, pois \(mdc(4,2) \neq 1\) e \(mdc(4,4) \neq 1\). Portanto, \(p = 2\) não é incluído na definição geral do teorema.
		
		Se estendermos a definição formalmente para \(p = 2\), o único termo válido seria quando \(n = 0\), ou seja,
		
		\[
		a_1 = \{4\}
		\]
		
		Por fim, considerando qualquer \(a_n \in B\), temos que todos os números pares da união das sequências \(A_i\) pertencem a \(A\), isto é,
		
		\[
		\bigcup_{n \in \mathbb{N}} a_n = A
		\]
		
		Encerrando, portanto, a demonstração do teorema.
	\end{proof}
	
	
	\subsection{Discussão dos Resultados}
	
	Apresente aqui a interpretação e análise dos resultados obtidos.
	
	%%%%%%%%%%%%%%%%%%%%%%%%%%%%%%%%%%%%%%%
	\section{Conclusão}
	
	Aqui você faz um resumo dos principais resultados, as contribuições do trabalho e possíveis direções para pesquisas futuras.
	
	%%%%%%%%%%%%%%%%%%%%%%%%%%%%%%%%%%%%%%%
	\section*{Agradecimentos}
	
	Se desejar, agradeça às instituições, agências de fomento ou colaboradores.
	
	%%%%%%%%%%%%%%%%%%%%%%%%%%%%%%%%%%%%%%%
	\begin{thebibliography}{9}
		
		\bibitem{apostol}
		T. M. Apostol.
		\textit{Introduction to Analytic Number Theory}.
		Springer, 1976.
		
		\bibitem{paiva}
		C. D. C. Paiva, et al. \textit{A busca pela prova da conjectura de Goldbach: explorando suas conquistas}. Caderno Pedagógico, 21(7): e5770-e5770, 2024.
		
		
		
		
	\end{thebibliography}
	%\bibliographystyle{amsplain} 
	%\bibliography{bio} 
	
\end{document}
